% Introduction chapter.
%
% Developed for my Master Thesis at Maastricht University.
% Based on Eugenio Senes's template at the University of Torino.
%
% By Joeri Hermans (joeri@joerihermans.com)
%
% Released under an MIT license. Share, modify and enjoy, but quote the author!

\chapter{Introduction}
\label{chapter:introduction}

In this chapter we introduce Distributed Deep Learning and the problems surrounding it. A more detailed description of the subject of study is given in Chapter~\ref{chapter:distributed_deep_learning}. Furthermore, we make the reader more comfortable with the notation and abbreviations used throughout this thesis. Finally, we formally define the problem statement in Section~\ref{sec:problem_statement}, and give an outline of this thesis in Section~\ref{sec:thesis_outline}.

\section[Distributed Deep Learning]{Distributed Deep Learning, an introduction}
\label{sec:intro_distributed_deep_learning}

Unsupervised feature learning and deep learning has shown that being able to train large models can drastically improve model performance. However, consider the problem of training a deep network with millions or even billions of parameters. How do we achieve this without waiting for days, or even weeks? Dean et al.~propose a different training paradigm which allows us to train and serve a model on multiple physical machines~\cite{dean2012large}. The authors propose two novel methodologies to distribute stochastic gradient descent.

\subsection{Model Parallelism}
\label{sec:intro_model_parallelism}

\subsection{Data Parallelism}
\label{sec:intro_data_parallelism}

\section{Problem Statement}
\label{sec:problem_statement}

\section{Thesis Outline}
\label{sec:thesis_outline}
