% Maastricht University Thesis Template
%
% Developed for my Master Thesis at Maastricht University.
% Based on Eugenio Senes's template at the University of Torino.
%
% By Joeri Hermans (joeri@joerihermans.com)
%
% Released under an MIT license. Share, modify and enjoy, but quote the author!

\documentclass[10pt, a4paper, oneside]{book}

% Define the packages.
% Packages are mostly based on the NIPS (Neural Information Processing Systems) requirements.
\usepackage[T1]{fontenc}              % Use 8-bit T1 fonts
\usepackage[backend=bibtex]{biblatex} % Citing
\usepackage[english]{babel}           % Set English as main language
\usepackage[intoc, english]{nomencl}  % Nomenclature
\usepackage[utf8]{inputenc}           % Allow utf-8 input
\usepackage{amsfonts}                 % Blackboard math symbols
\usepackage{amsmath}                  % AMS Math
\usepackage{amssymb}                  % AMS Symbols
\usepackage{appendix}                 % Appendix
\usepackage{booktabs}                 % Professional-quality tables
\usepackage{caption}                  % Captions
\usepackage{csquotes}                 % Context sensitive quotation facilities
\usepackage{float}                    % Float control
\usepackage{geometry}                 % Easily define margins
\usepackage{graphicx}                 % Graphic materials (e.g., images)
\usepackage{hyperref}                 % Hyperlinks
\usepackage{microtype}                % Microtypography
\usepackage{rotating}                 % Allow page rotation (e.g., for large table)
\usepackage{tikz}                     % Drawings
\usepackage{url}                      % Simple URL typesetting

% Define page structure using Geometry.
% For printing, set right to 35mm
\geometry{a4paper, portrait, left=35mm, right=20mm, top=35mm, bottom=30mm}

% Define the core properties of your thesis.
\title{Distributed Deep Learning}                                     % Title of the thesis
\author{Joeri R.~Hermans}                                             % Author name
\date{\today}                                                         % Publishing date
\def \university{Maastricht University}                               % University name
\def \universitycity{Maastricht}                                      % University city
\def \universitycountry{The Netherlands}                              % University country
\def \faculty{Faculty of Humanities and Sciences}                     % Faculty
\def \department{Department of Data Science \& Knowledge Engineering} % Department name

% Load the macros.
% Macro definitions for the Maastricht University Thesis Template.
%
% Developed for my Master Thesis at Maastricht University.
% Based on Eugenio Senes's template at the University of Torino.
%
% By Joeri Hermans (joeri@joerihermans.com)
%
% Released under an MIT license. Share, modify and enjoy, but quote the author!

\makeatletter

\newcommand{\makethesistitle}{
  % Draw the title.
  \begin{center}
    {\large \bf \textsc{\@title}}
  \end{center}
  % Draw the author.
  \center \small \@author
}

\newcommand{\makeinstitution}{
  \begin{center}
    \small
        {\university}\\
        {\faculty}\\
        {\department}\\
        {\universitycity, \universitycountry}
  \end{center}
}

\makeatother


% Use nice Tikz arrows.
\usetikzlibrary{arrows}
\usetikzlibrary{shapes.misc}
\usetikzlibrary{positioning}
\tikzset{>=latex}

% Add the bibliography database.
\addbibresource{bibliography.bib}

% Build the Nomenclature package.
\makenomenclature

\makeindex
\begin{document}

% Load the nomenclature.
% Maastricht University Thesis Template
%
% Developed for my Master Thesis at Maastricht University.
% Based on Eugenio Senes's template at the University of Torino.
%
% By Joeri Hermans (joeri@joerihermans.com)
%
% Released under an MIT license. Share, modify and enjoy, but quote the author!

% Define abbreviations.
\nomenclature{$\lambda$}{Communication period, or frequency of commits to the parameter server.}
\nomenclature{$\tilde{\theta}_t$}{Center variable, or central parametrization maintained by the parameter server.}
\nomenclature{$n$}{Number of parallel workers.}
\nomenclature{$\triangleq$}{Is defined as}
\nomenclature{$\tau$}{Staleness}
\nomenclature{$m$}{Mini-batch size}
\nomenclature{$J(\theta)$}{Loss with respect to parameterization $\theta$.}
\nomenclature{$\mathcal{L}(\theta~;~\textbf{x}~;~\textbf{y})$}{Loss function with respect to parametrization $\theta$, input $\textbf{x}$, and expected output $\textbf{y}$.}
\nomenclature{$\theta^k_t$}{Parametrization of worker $k$ at time $t$.}
\nomenclature{$\eta$}{Static learning rate}
\nomenclature{$\eta_t$}{Learning rate with respect to time $t$.}
\nomenclature{ADAG}{Asynchronous Distributed Adaptive Gradients}
\nomenclature{ASGD}{Asynchronous Stochastic Gradient Descent}
\nomenclature{CERN}{European Organization for Nuclear Research}
\nomenclature{CMS}{Compact Muon Solenoid}
\nomenclature{GD}{Gradient Descent}
\nomenclature{HEP}{High Energy Physics}
\nomenclature{EASGD}{Elastic Averaging Stochastic Gradient Descent}
\nomenclature{HL-LHC}{High Luminosity Large Hadron Collider}
\nomenclature{LHC}{Large Hadron Collider}
\nomenclature{MNIST}{Mixed National Institute of Standards and Technology database}
\nomenclature{PS}{Parameter Server}
\nomenclature{SGD}{Stochastic Gradient Descent}
\nomenclature{SE}{Statistical Efficiency}
\nomenclature{HE}{Hardware Efficiency}
\nomenclature{TE}{Temporal Efficiency}


% Start front matter.
\frontmatter
\let\cleardoublepage\clearpage
% Maastricht University Cover Page.
%
% Developed for my Master Thesis at Maastricht University.
% Based on Eugenio Senes's template at the University of Torino.
%
% By Joeri Hermans (joeri@joerihermans.com)
%
% Released under an MIT license. Share, modify and enjoy, but quote the author!

\makeatletter
\begin{titlepage}

  \vspace*{2cm}
  \center \Large Master Thesis
  \center \line(1,0){200}
  \vspace{0.1cm}
  \makethesistitle
  \vspace{0.1cm}
  \center \line(1,0){200}
  \vspace{0.3cm}
  \center \small Master Thesis \id
  \vspace{2cm}
  \begin{center}
    \small
    Thesis submitted in partial fulfillment of the requirements\\ for the degree of Master of Science of Artificial Intelligence
  \end{center}
  \vspace{.5cm}
  \begin{center}
        \small \textbf{Thesis Committee:}\\
        \committee
  \end{center}
  \vspace{.5cm}
  \makeinstitution
  \vspace{.5cm}
  \center \small \@date

\end{titlepage}
\makeatother

%% Maastricht University Dedication to / quote page
%
% Developed for my Master Thesis at Maastricht University.
% Based on Eugenio Senes's template at the University of Torino.
%
% By Joeri Hermans (joeri@joerihermans.com)
%
% Released under an MIT license. Share, modify and enjoy, but quote the author!

\newpage
\thispagestyle{empty}
\setcounter{page}{1}
\vspace*{6cm}
\begin{flushright}
The amazing quote\\
that I chose as inspiration\\
for this work\\
\vspace{4mm}
Author, \textit{Title}\\
\end{flushright}

% Maastricht University Preface
%
% Developed for my Master Thesis at Maastricht University.
% Based on Eugenio Senes's template at the University of Torino.
%
% By Joeri Hermans (joeri@joerihermans.com)
%
% Released under an MIT license. Share, modify and enjoy, but quote the author!

\newpage
\chapter*{Preface}
\addcontentsline{toc}{chapter}{Preface}

This thesis is submitted as a final requirement for the Master of Science degree at the Department of Data Science \& Knowledge Engineering of Maastricht University, The Netherlands. The subject of study originally started as a pilot project with Jean-Roch Vlimant, Maurizio Pierini, and Federico Presutti of the EP-UCM group (CMS experiment) at CERN. In order to handle the increased data rates of LHC Run 3 and High Luminosity LHC, the CMS experiment is considering to construct a new architecture for the High Level Trigger based on Deep Neural Networks. However, they would like to significantly decrease the training time of the models as well. This would allow them to tune the neural networks more frequently. As a result, we started to experiment with various state of the art distributed optimization algorithms. Which resulted in the achievements and insights presented in this thesis.\\

I would like to express my gratitude to several people. First and foremost, I would like to thank my promotors, Gerasimos Spanakis, and Rico M\"ockel for their expertise and suggestions during my research, which drastically improved the quality of this thesis. Furthermore, I would also like to thank my friends, colleagues and scientists at CERN for their support, feedback, and exchange of ideas during my stay there. It was a very motivating and inspiring time in my life. Especially the support and experience of my CERN supervisors, Zbigniew Baranowski, and Luca Canali, was proven to be invaluable on multiple occasions. I would also like to thank them for giving me the personal freedom to conduct my own research. Finally, I would like to thank my parents and grandparents who always supported me, and who gave me the chance to explore the world in this unique way.

\vspace{1cm}
\begin{flushright}
Joeri~R.~Hermans\\
Geneva, Switzerland 2016 - 2017
\end{flushright}

% Thesis abstract
%
% Developed for my Master Thesis at Maastricht University.
% Based on Eugenio Senes's template at the University of Torino.
%
% By Joeri Hermans (joeri@joerihermans.com)
%
% Released under an MIT license. Share, modify and enjoy, but quote the author!

\newpage
\chapter*{Abstract}
\addcontentsline{toc}{chapter}{Abstract}

Abstract here.

% Thesis summary.
%
% Developed for my Master Thesis at Maastricht University.
% Based on Eugenio Senes's template at the University of Torino.
%
% By Joeri Hermans (joeri@joerihermans.com)
%
% Released under an MIT license. Share, modify and enjoy, but quote the author!

\newpage
\chapter*{Summary}
\addcontentsline{toc}{chapter}{Summary}

Summary here.

\tableofcontents
\printnomenclature[3cm]

% Start main matter.
\mainmatter
% Introduction chapter.
%
% Developed for my Master Thesis at Maastricht University.
% Based on Eugenio Senes's template at the University of Torino.
%
% By Joeri Hermans (joeri@joerihermans.com)
%
% Released under an MIT license. Share, modify and enjoy, but quote the author!

\chapter[Introduction]{Introduction}
\label{chapter:introduction}

I cite myself \cite{distributed_keras}. This section will contain the introduction into the thesis problem.

% Chapter on Distributed Deep Learning.
%
% Developed for my Master Thesis at Maastricht University.
% Based on Eugenio Senes's template at the University of Torino.
%
% By Joeri Hermans (joeri@joerihermans.com)
%
% Released under an MIT license. Share, modify and enjoy, but quote the author!

\chapter{Distributed Deep Learning}
\label{chapter:distributed_deep_learning}


% Force the bibliography on a new page.
\clearpage
% Add the references to the table of contents.
\addcontentsline{toc}{chapter}{References}
% Draw the bibliography.
\printbibliography

% Start appendices.
\appendix
\appendixpage
\noappendicestocpagenum
\addappheadtotoc

\end{document}
